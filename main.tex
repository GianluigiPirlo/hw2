\documentclass[10pt]{article}
% math fonts
\usepackage{amsmath,amsfonts,amsthm,amssymb}
% to insert graphics
\usepackage{graphicx}
% to change margins of the pages
\usepackage[margin=0.9in]{geometry}

% Makes equations flush left
\usepackage{fleqn}


% This generates a page header with your name in it.
\usepackage{fancyhdr}
\pagestyle{fancy}
\fancyhf{}
\lhead{FOCS Fall 2018}
\rhead{HW02 solution by Jeff Trinkle}
\rfoot{Page \thepage}

% This package makes it easy to have boxes around large text.
\usepackage{framed}


\begin{document}


{\bf Rosen 1.3, Exercise 12(b):}  \\


\begin{proof}
\begin{framed}
If $[(p \rightarrow q)\land (q \rightarrow r)]$ is false then  must be true since $F \rightarrow T \equiv T$ and $F \rightarrow F \equiv T$\\
Assume $[(p \rightarrow q)\land (q \rightarrow r)]$ is true\\
Because $[(p \rightarrow q)\land (q \rightarrow r)]$ is true.\\
So $p \rightarrow q$ is true and $q \rightarrow r$ is true.\\
If p is false then $p \rightarrow r$ is true.
If p is true.\\
Because p is true.\\
$\neg p$ must be false and q must be true.\\
$\neg q$ must be false and r must be true.\\
so $p \rightarrow r$ is $T \rightarrow T \equiv T$\\

q.e.d. 
\end{framed}
\end{proof}



\newpage

\noindent
{\bf Rosen 1.3, Exercise 28:}  \\

\begin{proof}
\begin{framed}
\equiv $\big[(p \rightarrow q) \land (q \rightarrow p)]$\\
\equiv $(\neg p \lor q) \land (\neg q \lor p)$     DeMorgan's law\\
\equiv $(q \lor \neg p) \land (p \lor \neg q)$
commutative law\\
\equiv $(\neg \neg q \lor \neg p) \land (\neg \neg p \lor \neg q)$
double negation\\
\equiv $(\neg q \rightarrow \neg p) \land (\neg p \rightarrow \neg q)$
DeMorgan's law\\
\equiv $(\neg p \rightarrow \neg q) \land (\neg q \rightarrow \neg p)$
commutative law\\
\equiv \\
q.e.d.
\end{framed}
\end{proof}


\newpage

\noindent
{\bf Rosen 1.3, Exercise 62(c):}  \\

\begin{framed}
Put the answer here.
\end{framed}


\vspace*{1cm}
\noindent
Put the steps and arguments you used to arrive at your answer here.



\newpage


\noindent
{\bf Add-on :}  Put the compound proposition from Rosen1.3, Exercise 12(b) in disjunctive normal form.\\

\begin{framed}
final answer here
\end{framed}

\vspace*{1cm}
\noindent
Your work goes here


\newpage


\noindent
{\bf Rosen 1.4, Exercise 8(c):}  \\

\begin{framed}
final answer here
\end{framed}

\vspace*{1cm}
\noindent
Your work goes here


\newpage



\noindent
{\bf Rosen 1.4, Exercise 10(c):}  \\

\begin{framed}
final answer here
\end{framed}

\vspace*{1cm}
\noindent
Your work goes here


\newpage



\noindent
{\bf Rosen 1.4, Exercise 18(e):}  \\

\begin{framed}
final answer here
\end{framed}

\vspace*{1cm}
\noindent
Your work goes here


\newpage




\noindent
{\bf Rosen 1.4, Exercise 28(d):}  \\

\begin{framed}
final answer here
\end{framed}

\vspace*{1cm}
\noindent
Your work goes here


\newpage





\noindent
{\bf Rosen 1.4, Exercise 36(c):}  \\

\begin{framed}
final answer here
\end{framed}

\vspace*{1cm}
\noindent
Your work goes here


\newpage




\noindent
{\bf Rosen 1.5, Exercise 4(c):}  \\

\begin{framed}
final answer here
\end{framed}

\vspace*{1cm}
\noindent
Your work goes here


\newpage





\noindent
{\bf Rosen 1.5, Exercise 10(e):}  \\

\begin{framed}
final answer here
\end{framed}

\vspace*{1cm}
\noindent
Your work goes here


\newpage




\noindent
{\bf Rosen 1.5, Exercise 14(e):}  \\

\begin{framed}
final answer here
\end{framed}

\vspace*{1cm}
\noindent
Your work goes here


\newpage




\noindent
{\bf Rosen 1.5, Exercise 24(d):}  \\

\begin{framed}
final answer here
\end{framed}

\vspace*{1cm}
\noindent
Your work goes here


\newpage




\noindent
{\bf Rosen 1.5, Exercise 38(d):}  \\

\begin{framed}
final answer here
\end{framed}

\vspace*{1cm}
\noindent
Your work goes here






\end{document}

